% Options for packages loaded elsewhere
\PassOptionsToPackage{unicode}{hyperref}
\PassOptionsToPackage{hyphens}{url}
%
\documentclass[
  ignorenonframetext,
]{beamer}
\usepackage{pgfpages}
\setbeamertemplate{caption}[numbered]
\setbeamertemplate{caption label separator}{: }
\setbeamercolor{caption name}{fg=normal text.fg}
\beamertemplatenavigationsymbolsempty
% Prevent slide breaks in the middle of a paragraph
\widowpenalties 1 10000
\raggedbottom
\setbeamertemplate{part page}{
  \centering
  \begin{beamercolorbox}[sep=16pt,center]{part title}
    \usebeamerfont{part title}\insertpart\par
  \end{beamercolorbox}
}
\setbeamertemplate{section page}{
  \centering
  \begin{beamercolorbox}[sep=12pt,center]{part title}
    \usebeamerfont{section title}\insertsection\par
  \end{beamercolorbox}
}
\setbeamertemplate{subsection page}{
  \centering
  \begin{beamercolorbox}[sep=8pt,center]{part title}
    \usebeamerfont{subsection title}\insertsubsection\par
  \end{beamercolorbox}
}
\AtBeginPart{
  \frame{\partpage}
}
\AtBeginSection{
  \ifbibliography
  \else
    \frame{\sectionpage}
  \fi
}
\AtBeginSubsection{
  \frame{\subsectionpage}
}

\usepackage{amsmath,amssymb}
\usepackage{iftex}
\ifPDFTeX
  \usepackage[T1]{fontenc}
  \usepackage[utf8]{inputenc}
  \usepackage{textcomp} % provide euro and other symbols
\else % if luatex or xetex
  \usepackage{unicode-math}
  \defaultfontfeatures{Scale=MatchLowercase}
  \defaultfontfeatures[\rmfamily]{Ligatures=TeX,Scale=1}
\fi
\usepackage{lmodern}
\usetheme[]{Berlin}
\ifPDFTeX\else  
    % xetex/luatex font selection
\fi
% Use upquote if available, for straight quotes in verbatim environments
\IfFileExists{upquote.sty}{\usepackage{upquote}}{}
\IfFileExists{microtype.sty}{% use microtype if available
  \usepackage[]{microtype}
  \UseMicrotypeSet[protrusion]{basicmath} % disable protrusion for tt fonts
}{}
\makeatletter
\@ifundefined{KOMAClassName}{% if non-KOMA class
  \IfFileExists{parskip.sty}{%
    \usepackage{parskip}
  }{% else
    \setlength{\parindent}{0pt}
    \setlength{\parskip}{6pt plus 2pt minus 1pt}}
}{% if KOMA class
  \KOMAoptions{parskip=half}}
\makeatother
\usepackage{xcolor}
\newif\ifbibliography
\setlength{\emergencystretch}{3em} % prevent overfull lines
\setcounter{secnumdepth}{-\maxdimen} % remove section numbering


\providecommand{\tightlist}{%
  \setlength{\itemsep}{0pt}\setlength{\parskip}{0pt}}\usepackage{longtable,booktabs,array}
\usepackage{calc} % for calculating minipage widths
\usepackage{caption}
% Make caption package work with longtable
\makeatletter
\def\fnum@table{\tablename~\thetable}
\makeatother
\usepackage{graphicx}
\makeatletter
\def\maxwidth{\ifdim\Gin@nat@width>\linewidth\linewidth\else\Gin@nat@width\fi}
\def\maxheight{\ifdim\Gin@nat@height>\textheight\textheight\else\Gin@nat@height\fi}
\makeatother
% Scale images if necessary, so that they will not overflow the page
% margins by default, and it is still possible to overwrite the defaults
% using explicit options in \includegraphics[width, height, ...]{}
\setkeys{Gin}{width=\maxwidth,height=\maxheight,keepaspectratio}
% Set default figure placement to htbp
\makeatletter
\def\fps@figure{htbp}
\makeatother
\newlength{\cslhangindent}
\setlength{\cslhangindent}{1.5em}
\newlength{\csllabelwidth}
\setlength{\csllabelwidth}{3em}
\newlength{\cslentryspacingunit} % times entry-spacing
\setlength{\cslentryspacingunit}{\parskip}
\newenvironment{CSLReferences}[2] % #1 hanging-ident, #2 entry spacing
 {% don't indent paragraphs
  \setlength{\parindent}{0pt}
  % turn on hanging indent if param 1 is 1
  \ifodd #1
  \let\oldpar\par
  \def\par{\hangindent=\cslhangindent\oldpar}
  \fi
  % set entry spacing
  \setlength{\parskip}{#2\cslentryspacingunit}
 }%
 {}
\usepackage{calc}
\newcommand{\CSLBlock}[1]{#1\hfill\break}
\newcommand{\CSLLeftMargin}[1]{\parbox[t]{\csllabelwidth}{#1}}
\newcommand{\CSLRightInline}[1]{\parbox[t]{\linewidth - \csllabelwidth}{#1}\break}
\newcommand{\CSLIndent}[1]{\hspace{\cslhangindent}#1}

\makeatletter
\makeatother
\makeatletter
\makeatother
\makeatletter
\@ifpackageloaded{caption}{}{\usepackage{caption}}
\AtBeginDocument{%
\ifdefined\contentsname
  \renewcommand*\contentsname{Table of contents}
\else
  \newcommand\contentsname{Table of contents}
\fi
\ifdefined\listfigurename
  \renewcommand*\listfigurename{List of Figures}
\else
  \newcommand\listfigurename{List of Figures}
\fi
\ifdefined\listtablename
  \renewcommand*\listtablename{List of Tables}
\else
  \newcommand\listtablename{List of Tables}
\fi
\ifdefined\figurename
  \renewcommand*\figurename{Figure}
\else
  \newcommand\figurename{Figure}
\fi
\ifdefined\tablename
  \renewcommand*\tablename{Table}
\else
  \newcommand\tablename{Table}
\fi
}
\@ifpackageloaded{float}{}{\usepackage{float}}
\floatstyle{ruled}
\@ifundefined{c@chapter}{\newfloat{codelisting}{h}{lop}}{\newfloat{codelisting}{h}{lop}[chapter]}
\floatname{codelisting}{Listing}
\newcommand*\listoflistings{\listof{codelisting}{List of Listings}}
\makeatother
\makeatletter
\@ifpackageloaded{caption}{}{\usepackage{caption}}
\@ifpackageloaded{subcaption}{}{\usepackage{subcaption}}
\makeatother
\makeatletter
\@ifpackageloaded{tcolorbox}{}{\usepackage[skins,breakable]{tcolorbox}}
\makeatother
\makeatletter
\@ifundefined{shadecolor}{\definecolor{shadecolor}{rgb}{.97, .97, .97}}
\makeatother
\makeatletter
\makeatother
\makeatletter
\makeatother
\ifLuaTeX
  \usepackage{selnolig}  % disable illegal ligatures
\fi
\IfFileExists{bookmark.sty}{\usepackage{bookmark}}{\usepackage{hyperref}}
\IfFileExists{xurl.sty}{\usepackage{xurl}}{} % add URL line breaks if available
\urlstyle{same} % disable monospaced font for URLs
\hypersetup{
  pdftitle={Class 1 - Research I: Principles},
  hidelinks,
  pdfcreator={LaTeX via pandoc}}

\title{Class 1 - Research I: Principles}
\author{}
\date{}

\begin{document}
\frame{\titlepage}
\ifdefined\Shaded\renewenvironment{Shaded}{\begin{tcolorbox}[enhanced, boxrule=0pt, interior hidden, breakable, frame hidden, borderline west={3pt}{0pt}{shadecolor}, sharp corners]}{\end{tcolorbox}}\fi

\begin{frame}{Agenda}
\protect\hypertarget{agenda}{}
\begin{itemize}
\item
  Introductions (30 minutes)

  \begin{itemize}
  \tightlist
  \item
    Getting to know each other
  \item
    Syllabus and materials overview
  \item
    Typical class flow
  \end{itemize}
\item
  Readings for today (70 minutes, with 5 min break)
\item
  Summative lecture and open discussion (20 minutes)

  \begin{itemize}
  \tightlist
  \item
    Key principles
  \item
    Additional thoughts
  \end{itemize}
\end{itemize}
\end{frame}

\hypertarget{introductions}{%
\section{Introductions}\label{introductions}}

\begin{frame}{A little about me}
\protect\hypertarget{a-little-about-me}{}
\begin{figure}

{\centering \includegraphics{assets/alittleaboutme.png}

}

\caption{A little about me}

\end{figure}
\end{frame}

\begin{frame}{A little about me}
\protect\hypertarget{a-little-about-me-1}{}
\includegraphics{assets/alittleaboutme1.png}
\end{frame}

\begin{frame}{A little about me}
\protect\hypertarget{a-little-about-me-2}{}
\begin{columns}[T]
\begin{column}{0.45\textwidth}
\includegraphics{assets/alittleaboutme3.jpeg}
\end{column}

\begin{column}{0.45\textwidth}
\includegraphics{assets/alittleaboutme4.jpeg}
\end{column}
\end{columns}
\end{frame}

\begin{frame}{A little about you}
\protect\hypertarget{a-little-about-you}{}
Let's fill out some introductory surveys:
\href{https://PollEv.com/drfox}{Pollev.com/drfox}
\end{frame}

\begin{frame}{Syllabus and materials overview}
\protect\hypertarget{syllabus-and-materials-overview}{}
\begin{itemize}
\tightlist
\item
  \href{https://briancfox.com/ResearchDesign/docs/syllabus.html}{Syllabus}
\item
  \href{https://brightspace.bentley.edu}{Brightspace}
\item
  \href{https://www.dropbox.com/scl/fo/ny0hiyu93bep31udth6ot/h?rlkey=or2mfwzqbdzkwbv8ky4yws5h7\&dl=0}{Dropbox}
\end{itemize}
\end{frame}

\begin{frame}{Typical class flow}
\protect\hypertarget{typical-class-flow}{}
\begin{itemize}
\item
  \emph{Part I:} Conceptual grounding and agenda setting (10 minutes)
\item
  \emph{Part II:} Core paper discussion (45 minutes):

  \begin{itemize}
  \item
    We will discuss the 2-3 papers that all students have been assigned
    to read in detail
  \item
    These papers typically will provide a mix of conceptual background
    and how-to guides
  \end{itemize}
\item
  \emph{Break}
\end{itemize}
\end{frame}

\begin{frame}{Typical class flow}
\protect\hypertarget{typical-class-flow-1}{}
\begin{itemize}
\item
  \emph{Part III:} Activity period (40 minutes):

  \begin{itemize}
  \item
    (Weeks 2 -- 7) Compare / contrast: One group tasked with reviewing
    two additional papers to explain their points of intersection,
    divergence, and ties to core papers
  \item
    (Weeks 8 -- 14) Replication: One group tasked with using data from
    one of my current or published papers to replicate analyses and show
    the class the process
  \end{itemize}
\end{itemize}
\end{frame}

\begin{frame}{Typical class flow}
\protect\hypertarget{typical-class-flow-2}{}
\begin{itemize}
\item
  \emph{Part IV:} Summative lecture on concepts (15 minutes):

  \begin{itemize}
  \item
    I will make a brief presentation to tie together and highlight key
    concepts
  \item
    Elements missed in the general discussion will be given greater
    focus
  \end{itemize}
\item
  \emph{Part V:} Open discussion (5 minutes)
\end{itemize}
\end{frame}

\hypertarget{readings-for-today}{%
\section{Readings for Today}\label{readings-for-today}}

\begin{frame}{Preamble}
\protect\hypertarget{preamble}{}
I have provided some discussion questions for us to consider in case we
need to get the ball rolling.

We may or may not discuss those questions depending on the flow of the
class.

In general, I would rather talk about your ideas and questions rather
than these ``canned'' items.
\end{frame}

\begin{frame}{Readings}
\protect\hypertarget{readings}{}
\begin{enumerate}
\item
  Popper, K. R. (2002). The Logic of Scientific Discovery. Routledge.
  {[}Ch .1{]}
\item
  Mantere, S., \& Ketokivi, M. 2013. Reasoning in Organization Science.
  Academy of Management Review, 38(1), 70-89.
\item
  Nosek, B. A. \& Errington, T. M. 2020. What is replication? PLOS
  Biology: 1-8.
\item
  Rynes, S. L., \& Bartunek, J. M. (2017). Evidence-Based Management:
  Foundations, Development, Controversies and Future. Annual Review of
  Organizational Psychology and Organizational Behavior, 4(1), 235-261.
\end{enumerate}
\end{frame}

\begin{frame}{Popper (2002)}
\protect\hypertarget{popper-2002}{}
The Logic of Scientific Discovery. {[}Ch .1{]}

\begin{quote}
According to the view that will be put forward here, the method of
critically testing theories, and selecting them according to the results
of tests, always proceeds on the following lines. From a new idea, put
up tentatively, and not yet justified in any way---an anticipation, a
hypoth- esis, a theoretical system, or what you will---conclusions are
drawn by means of logical deduction {[}\ldots{]}
\end{quote}

\begin{quote}
{[}Then,{]} there is the testing of the theory by way of empirical
applications of the conclusions which can be derived from it. {[}p.~9{]}
\end{quote}
\end{frame}

\begin{frame}{Popper (2002)}
\protect\hypertarget{popper-2002-1}{}
\begin{columns}[T]
\begin{column}{0.7\textwidth}
Discussion Questions

\begin{itemize}
\item
  Reactions? Insights? Disagreements?
\item
  In your view, what is the main point?
\item
  Do this worldview currently inform your work? How might it?
\end{itemize}
\end{column}

\begin{column}{0.3\textwidth}
\begin{figure}

{\centering 

\href{https://www.dropbox.com/scl/fi/45sn8qzoozrk78ecqhkm3/Popper-2002-16883.pdf?rlkey=gydn6gc0goixtyg6sglv9tpop\&dl=0}{\includegraphics{assets/Popper 2002.png}}

}

\end{figure}
\end{column}
\end{columns}
\end{frame}

\begin{frame}{Mantere and Ketokivi (2013)}
\protect\hypertarget{mantere-and-ketokivi-2013}{}
Reasoning in Organization Science. Academy of Management Review, 38(1),
70-89.

\begin{quote}
Labels aside, a closer look at research practice reveals that
researchers across research traditions use all three forms of reasoning.
It is hardly surprising to observe that we all make inferences to a case
(use deduction), inferences to generalizations (use induction), and
infer ences to explanations (use abduction). Thus, using reasoning types
as labels to describe entire research designs is misleading. Instead,
differences between research approaches, whatever they may be, are found
not in the types of rea soning used but, rather, in how the three
reasoning types are used in conjunction with one an other. (p.~76)
\end{quote}
\end{frame}

\begin{frame}{Mantere and Ketokivi (2013)}
\protect\hypertarget{mantere-and-ketokivi-2013-1}{}
\begin{columns}[T]
\begin{column}{0.7\textwidth}
Discussion Questions

\begin{itemize}
\item
  What the hell are they talking about?
\item
  What mode(s) of reasoning do you tend to rely on in your current work?
\item
  What concrete practices did you draw from this paper, if any?
\end{itemize}
\end{column}

\begin{column}{0.3\textwidth}
\begin{figure}

{\centering 

\href{https://www.dropbox.com/scl/fi/xvorqwk6u47kp49z5tb7y/Mantere-and-Ketokivi-2013-47636.pdf?rlkey=u303k8llz8qma053nchn7ncfb\&dl=0}{\includegraphics{assets/Mantere and Ketokivi 2013.png}}

}

\end{figure}
\end{column}
\end{columns}
\end{frame}

\begin{frame}{Nosek and Errington (2020)}
\protect\hypertarget{nosek-and-errington-2020}{}
What is replication? PLOS Biology: 1-8.

\begin{quote}
To be a replication, 2 things must be true: outcomes consistent with a
prior claim would increase confidence in the claim, and outcomes
inconsistent with a prior claim would decrease confidence in the claim.
The symmetry promotes replication as a mechanism for confronting prior
claims with new evidence. Therefore, declaring that a study is a
replication is a theoretical commitment. Replication provides the
opportunity to test whether existing theories, hypothe- ses, or models
are able to predict outcomes that have not yet been observed. Successful
replica- tions increase confidence in those models; unsuccessful
replications decrease confidence and spur theoretical innovation to
improve or discard the model. (p.~2)
\end{quote}
\end{frame}

\begin{frame}{Nosek and Errington (2020)}
\protect\hypertarget{nosek-and-errington-2020-1}{}
\begin{columns}[T]
\begin{column}{0.7\textwidth}
Discussion Questions

\begin{itemize}
\item
  Do you agree with their definition of replication?

  \begin{itemize}
  \item
    What are the benefits and drawbacks of applying such a definition?
  \item
    How does this fit in with the
    \href{https://en.wikipedia.org/wiki/Replication_crisis}{replication
    crisis}?
  \end{itemize}
\end{itemize}
\end{column}

\begin{column}{0.3\textwidth}
\begin{figure}

{\centering 

\href{https://www.dropbox.com/scl/fi/30uiw2vtvsk6fvfnkgn0j/Nosek-and-Errington-2020-236800.pdf?rlkey=mnlwdyu57el5ufdmjwd6izx4j\&dl=0}{\includegraphics{assets/Nosek and Errington 2020.png}}

}

\end{figure}
\end{column}
\end{columns}
\end{frame}

\begin{frame}{Rynes and Bartunek (2017)}
\protect\hypertarget{rynes-and-bartunek-2017}{}
Evidence-Based Management: Foundations, Development, Controversies and
Future.

\begin{quote}
Management academics have long noted a large gap between academic
research and managerial practice. {[}\ldots{]} Some have viewed the
causes of the gap as lying primarily with academic researchers, who are
characterized (perhaps caricatured) as having become overspecialized,
self-referential, obsessed with theory, excessively mathematical,
jargonladen, unconcerned about practical problems, and dismissive of
practitioners {[}\ldots{]} Others have focused on practitioners, who are
sometimes characterized or caricatured as research phobic,
anti-intellectual, susceptible to unproven fads and fashions\ldots{}
(p.~236)
\end{quote}
\end{frame}

\begin{frame}{Rynes and Bartunek (2017)}
\protect\hypertarget{rynes-and-bartunek-2017-1}{}
\begin{columns}[T]
\begin{column}{0.7\textwidth}
Discussion Questions

\begin{itemize}
\item
  Are you familiar with evidence-based practice from your current work?
\item
  In your PhD studies so far, have you seen a concerted effort to move
  towards evidence-based management?
\item
  Where might you fit in helping to advance evidence-based management?
  How might you do about doing it?
\end{itemize}
\end{column}

\begin{column}{0.3\textwidth}
\begin{figure}

{\centering 

\href{https://www.dropbox.com/scl/fi/71tkfjk6fdbv69jnllc28/Rynes-and-Bartunek-2017-42.pdf?rlkey=p421n3a286wmwa90bpzc7uryi\&dl=0}{\includegraphics{assets/Rynes and Bartunek 2017.png}}

}

\end{figure}
\end{column}
\end{columns}
\end{frame}

\hypertarget{summative-lecture}{%
\section{Summative lecture}\label{summative-lecture}}

\begin{frame}{Preamble}
\protect\hypertarget{preamble-1}{}
What follows is my personal, idiosyncratic synthesis of the pieces that
we have read to date. To be clear, many interpretations are possible due
to these articles' collective:

\begin{itemize}
\tightlist
\item
  richness
\item
  overlap
\item
  distinctive features
\end{itemize}

Furthermore, there are multiple plausible criteria to judge quality
research and a lack of universal consensus given the multiplicity of
aims and epistemological orientations.

This is not to say anything goes; rather, I am trying to highlight the
limits of my knowledge and my unique lens that necessarily abstracts
away features from complex topics.
\end{frame}

\begin{frame}{Some key principles of research design}
\protect\hypertarget{some-key-principles-of-research-design}{}
\begin{itemize}
\item
  Falsifiability
\item
  Defensibility
\item
  Applicability
\item
  Replicability
\end{itemize}
\end{frame}

\begin{frame}{Falsifiability}
\protect\hypertarget{falsifiability}{}
\begin{itemize}
\tightlist
\item
  Falsifiability provides a basis for to use abductive reasoning to
  augment pure deductive reasoning. The latter is true a priori if the
  premises and statements are valid. Thus, pure deduction an transform
  our understanding of the system, but cannot generate truth from
  outside the established system - this must be uncovered by other
  means.
\end{itemize}
\end{frame}

\begin{frame}{Falsifiability}
\protect\hypertarget{falsifiability-1}{}
\begin{quote}
My proposal is based upon an asymmetry between verifiability and
falsifiability; an asymmetry which results from the logical form of
universal statements. For these are never derivable from singular
statements, but can be contradicted by singular statements. - Popper
(2002, 19)
\end{quote}
\end{frame}

\begin{frame}{Defensibility}
\protect\hypertarget{defensibility}{}
\begin{itemize}
\item
  If the logic of our arguments are defensible and the evidentiary basis
  is sound, we are better able to act upon the conclusions with
  confidence.
\item
  We rely on multiple modes of inference to assert our claims
  credibility - my coauthors and I argue that the relative importance of
  each depends on the type of research design employed and intended
  contribution.
\end{itemize}
\end{frame}

\begin{frame}{Defensibility}
\protect\hypertarget{defensibility-1}{}
\begin{columns}[T]
\begin{column}{0.9\textwidth}
\begin{figure}

{\centering \includegraphics{assets/Simsek and Fox WIP.png}

}

\caption{\label{fig-defensibility}The relative importance of inference
modes by type of research design}

\end{figure}
\end{column}
\end{columns}
\end{frame}

\begin{frame}{Defensibility}
\protect\hypertarget{defensibility-2}{}
\begin{columns}[T]
\begin{column}{0.48\textwidth}
One way to map out the defensibility of an argument is with a Toulmin
diagram:

\begin{figure}

{\centering 

\href{https://owl.purdue.edu/owl/general_writing/academic_writing/historical_perspectives_on_argumentation/toulmin_argument.html}{\includegraphics{assets/toulmin.png}}

}

\caption{An example Toulmin diagram}

\end{figure}
\end{column}
\end{columns}
\end{frame}

\begin{frame}{Applicability}
\protect\hypertarget{applicability}{}
But our arguments and conclusions, even if correct, are irrelevant if
they aren't applicable to real-world problems.

There are two corollaries to this:

\begin{itemize}
\item
  We should pick problems that actually matter, not just intellectual
  curiosities.
\item
  We should not be hamstrung by our ability to tackle important
  problems.
\end{itemize}
\end{frame}

\begin{frame}{Applicability}
\protect\hypertarget{applicability-1}{}
\begin{quote}
I was recently at a brown-bag seminar where a pair of management
colleagues were seeking advice about a preliminary research idea. It
took just a few minutes for us all to agree that their research ques-
tion was fascinating. It addressed an extremely in- teresting issue that
both academics and practicing managers would like to learn more about.
The only problem: the presenters had no theory. So, we spent the entire
session going through our collec- tive mental catalogues of theories
that might be invoked so that the project could proceed and have some
prospect of publication. People were men- tioning theories I'd never
heard of. We became frenzied, nearly desperate: ``Good god, there must
be a theory out there that we can latch onto.'' - Hambrick (2007)
\end{quote}
\end{frame}

\begin{frame}{Replicability}
\protect\hypertarget{replicability}{}
\begin{itemize}
\tightlist
\item
  Finally, the structure of our empirical base presumes that the
  research was performed in good order and that the findings are
  replicable within their domain of applicability.
\end{itemize}
\end{frame}

\begin{frame}{Replicability}
\protect\hypertarget{replicability-1}{}
\begin{quote}
{[}A{]}n accumulation of evidence that points to empirical regularities
provides us with a much broader and more generalized understanding of
the world. Such empirical regularities are known as `stylized facts'. -
Helfat (2007)
\end{quote}
\end{frame}

\begin{frame}{The relative importance of each principle}
\protect\hypertarget{the-relative-importance-of-each-principle}{}
We can consider four basic ``classes'' of research in management:

\begin{itemize}
\item
  basic disciplinary research (primary studies in AER, AJS)
\item
  applied research conducted in a management context (primary studies in
  AMJ)
\item
  data-driven decision making derived from primary studies (systematic
  reviews in IJMR, JOM)
\item
  practitioner-focused outlets (articles in HBR, CMR, popular press)
\end{itemize}

Where might, for example, applicability be more highly valued?
Falsifiability?
\end{frame}

\begin{frame}{Other thoughts: Useful types of thinking when conducting
research}
\protect\hypertarget{other-thoughts-useful-types-of-thinking-when-conducting-research}{}
\begin{itemize}
\item
  Skeptical thinking
\item
  Bayesian thinking
\item
  Strategic thinking
\item
  First principles thinking
\end{itemize}
\end{frame}

\begin{frame}{Skeptical thinking}
\protect\hypertarget{skeptical-thinking}{}
\begin{quote}
``Science depends on organized skepticism, that is, on continual,
methodical doubting. Few of us doubt our own conclusions, so science
embraces its skeptical approach by rewarding those who doubt someone
else's.'' ― Neil deGrasse Tyson, Origins: Fourteen Billion Years of
Cosmic Evolution
\end{quote}
\end{frame}

\begin{frame}{Skeptical thinking}
\protect\hypertarget{skeptical-thinking-1}{}
\begin{figure}

{\centering 

\href{https://www.youtube.com/watch?v=io6QdGcoWMU}{\includegraphics{assets/skepticism.png}}

}

\caption{Tyson on Skepticism}

\end{figure}
\end{frame}

\begin{frame}{Bayesian thinking}
\protect\hypertarget{bayesian-thinking}{}
Implicit in the discussions above is a question of degree of belief.

\begin{itemize}
\item
  Nosek and Errington talk about how replication increases or decreases
  our degree of belief.
\item
  Popper uses the asymmetry of verification to achieve binary outcome of
  disconfirmed evidence.
\item
  But couldn't we be more subtle in our treatment of beliefs?

  \begin{itemize}
  \item
    Indeed, we can through the application of Bayesian logic and Bayes'
    Rule.
  \item
    I will not be teaching you the statistical methods that follow from
    this, but you can find them.
  \end{itemize}
\end{itemize}
\end{frame}

\begin{frame}{Bayesian thinking}
\protect\hypertarget{bayesian-thinking-1}{}
\begin{figure}

{\centering 

\href{https://www.youtube.com/watch?v=HZGCoVF3YvM}{\includegraphics{assets/bayes.png}}

}

\caption{A Primer on Bayesian Thinking}

\end{figure}
\end{frame}

\begin{frame}{Strategic thinking}
\protect\hypertarget{strategic-thinking}{}
Finally, it helps to be strategic when thinking about designing and
evaluating research. By this I mean thinking that embraces three
characteristics:

\begin{itemize}
\item
  Rigor
\item
  Complexity
\item
  Ambiguity
\end{itemize}
\end{frame}

\begin{frame}{Strategic thinking}
\protect\hypertarget{strategic-thinking-1}{}
\begin{itemize}
\item
  Rigor

  \begin{itemize}
  \item
    Comprehensive -- focusing attention on both the forest (a research
    program) and the trees (discrete methods or studies)
  \item
    Adaptive -- balancing multiple goals and knowing what progress can
    be made against one or more of them simultaneously
  \item
    Inferential -- moving from what is known to what can be reasonably
    inferred
  \end{itemize}
\end{itemize}
\end{frame}

\begin{frame}{Strategic thinking}
\protect\hypertarget{strategic-thinking-2}{}
\begin{itemize}
\item
  Complexity

  \begin{itemize}
  \item
    Dynamics -- accounting for first and second order effects that are
    material across actors, choices, and time
  \item
    Allocentricity -- outcomes often jointly determined by internal and
    external factors, often other parties or agents
  \end{itemize}
\end{itemize}
\end{frame}

\begin{frame}{Strategic thinking}
\protect\hypertarget{strategic-thinking-3}{}
\begin{itemize}
\item
  Ambiguity

  \begin{itemize}
  \item
    Unstable -- non-linear shifts across time and situations may limit
    generalizability and heighten the role of context
  \item
    Unforeseeable -- many research projects are a full reinforcement
    learning problem, learning while doing is necessary to reveal the
    evolving state of the world
  \end{itemize}
\end{itemize}
\end{frame}

\begin{frame}{Strategic thinking}
\protect\hypertarget{strategic-thinking-4}{}
\begin{figure}

{\centering 

\href{https://www.youtube.com/watch?v=AkgsYA-LYxo}{\includegraphics{assets/bremmer.png}}

}

\caption{Bremmer's view of strategic thinking}

\end{figure}
\end{frame}

\begin{frame}{First principles thinking}
\protect\hypertarget{first-principles-thinking}{}
\begin{quote}
A first principle is a basic assumption that cannot be deduced any
further. Over two thousand years ago, Aristotle defined a first
principle as ``the first basis from which a thing is known.'' First
principles thinking is a fancy way of saying ``think like a scientist.''
Scientists don't assume anything. They start with questions like, What
are we absolutely sure is true? What has been proven? - James Clear
\end{quote}
\end{frame}

\hypertarget{preparation-for-next-class}{%
\section{Preparation for Next Class}\label{preparation-for-next-class}}

\begin{frame}{Next class}
\protect\hypertarget{next-class}{}
\textbf{Research II: Positions}

\begin{enumerate}
\item
  Huff, A. S. (1999). Writing for Scholarly Publication. SAGE. {[}Chs.
  1, 3 {]}
\item
  McGrath, Joseph E. (1981) Dilemmatics: The Study of Research Choices
  and Dilemmas, American Behavioral Scientist, 25, 2, 179-210
\item
  Simsek, Z., Heavey, C., Fox, B. C., \& Yu, T. 2022. Compelling
  Questions in Research: Seeing What Everybody Has Seen and Thinking
  What Nobody Has Thought. Journal of Management, 48(6), 1347-1365.
\end{enumerate}
\end{frame}

\begin{frame}{Next class}
\protect\hypertarget{next-class-1}{}
\textbf{Research II: Positions}

Our first compare and contrast discussion will take place.

Presenters, please reach out if you have questions or concerns!

\begin{enumerate}
\setcounter{enumi}{3}
\tightlist
\item
  Compare / Contrast
\end{enumerate}

\begin{itemize}
\item
  Daft, R. L., \& Lewin, A. Y. 2008. Rigor and relevance in organization
  studies: Idea migration and academic journal evolution. Organization
  Science, 19: 177-183.
\item
  Tushman, M., \& O'Reilly, C. (2007). Research and Relevance:
  Implications of Pasteur's Quadrant for Doctoral Programs and Faculty
  Development. The Academy of Management Journal, 50, No.~4, 769-774.
  \url{https://doi.org/10.2307/20159888}
\end{itemize}
\end{frame}

\begin{frame}{References}
\protect\hypertarget{references}{}
\hypertarget{refs}{}
\begin{CSLReferences}{1}{0}
\leavevmode\vadjust pre{\hypertarget{ref-hambrick2007}{}}%
Hambrick, Donald C. 2007. {``The Field of Management's Devotion to
Theory: Too Much of a Good Thing.''} \emph{The Academy of Management
Journal} 50, No. 6: 1346--52.

\leavevmode\vadjust pre{\hypertarget{ref-helfat2007}{}}%
Helfat, Constance E. 2007. {``Stylized Facts, Empirical Research and
Theory Development in Management.''} \emph{Strategic Organization} 5
(2): 185--92.

\leavevmode\vadjust pre{\hypertarget{ref-popper2002}{}}%
Popper, Karl R. 2002. \emph{The Logic of Scientific Discovery}.
Routledge.

\end{CSLReferences}
\end{frame}



\end{document}
